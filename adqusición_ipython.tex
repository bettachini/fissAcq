\documentclass[a4paper]{article}

\usepackage[margin=2cm]{geometry}
\usepackage[utf8]{inputenc}
\usepackage[locale=FR]{siunitx}
\def\spanishoptions{argentina}
\usepackage[spanish]{babel}

\begin{document}


\title{Adquisición desde osciloscopio Tex TDS 2002B}
\date{\today}
\author{Víctor A. Bettachini}
\maketitle

\section{USBTMC}
basado en \verb'http://www.cibomahto.com/2010/04/controlling-a-rigol-oscilloscope-using-linux-and-python/'

@ TDS 2002B
Encendido
utilidades/Puerto USB trasero -> computadora
figura bajo /dev/usbtmc0

@ tek.py están las clases usbtmc y tek

Uso como base código \verb'https://pypi.python.org/pypi/PyTektronixScope' que usa PyVISA


\subsection{permisos usbtmc0}
Corregido el problema de permisos con el puerto usb
USBError: [Errno 13] Access denied (insufficient permissions)

Creo una regla de udev en \verb'/lib/udev/rules.d/40-usbtmc-permissions.rules'
para el módulo del kernel de usbtmc según instrucciones en \verb'www.raspberrypi.org/forums/viewtopic.php?t=60784&p=485926
'. 

\begin{verbatim}
# à la https://www.raspberrypi.org/forums/viewtopic.php?t=60784&p=485926
# Devices
KERNEL=="usbtmc/*", MODE="0660", GROUP="dialout"
KERNEL=="usbtmc[0-9]*", MODE="0660", GROUP="dialout"
\end{verbatim}

Donde el MODE lo obtuve de \verb'stat -c %a /dev/ttyS0' que corresponde a \verb'crw-rw----'
y GROUP es \verb'dialout' al igual que el dispositivo anterior.



No funcionó lo propuesto en \verb'http://svn.openmoko.org/developers/werner/ahrt/host/tmc/etc/40-usbtmc-permissions.rules'
De lsusb obtengo \verb'Bus 002 Device 005: ID 0699:0364 Tektronix, Inc.'
por lo que en \verb'/lib/udev/rules.d/40-usbtmc-permissions.rules' debiera figurar 
\verb'ATTRS{idVendor}=="0699", ATTRS{idProduct}=="0346", MODE="0660", GROUP="dialout"'


Pero ahora tengo un problema de timeout.
Quito lo regla y vuelvo a darle permisos útiles manualmente.

\begin{verbatim}
sudo chmod --reference /dev/ttyS0 /dev/usbtmc0
sudo chgrp dialout /dev/usbtmc0
\end{verbatim}



\section{Error de timeout}
\verb'OSError: [Errno 110] Connection timed out'
Causado parque el canal que se pide transferir no está activo/ no tiene datos.

Con \verb'DATa:SOUrce?' ví que el canal a transferir era el 2, pero estaba sin datos.
Bastó con establecer \verb'DATa:SOUrce CH1' para que todo volviera a la normalidad. 


\section{PyVISA}

Los errores con Python2 fueron demasiados. \ref{Python2}
Pero ahora tengo el mismo error de time out. 


\section{Python3}
sudo apt-get ipython3
sudo -HE python3 get-pip.py at no avail... sudo apt-get install python3-pip
sudo -HE pip-3.2 install -U pyvisa
sudo -EH pip-3.2 install pyvisa-py
sudo -EH pip-3.2 install pyserial
sudo -EH pip-3.2 install pyusb

Con python3 evité los errores anteriores y pude hacer un *IDN? corriendo bajo sudo -E ipython3


\subsubsection{Corrida típica}
@ TDS 2002B
Encendido
utilidades/Puerto USB trasero -> Ordenador

\begin{verbatim}
In [1]: import visa

In [2]: rm = visa.ResourceManager('@py')

In [3]: print(rm.list_resources())
('ASRL/dev/ttyS1::INSTR', 'ASRL/dev/ttyS0::INSTR', 'ASRL/dev/ttyS3::INSTR', 'ASRL/dev/ttyS2::INSTR', 'USB0::1689::868::C101891::0::INSTR')

In [4]: my_instrument = rm.open_resource('USB0::1689::868::C101891::0::INSTR')

In [5]: print(my_instrument.query("*IDN?") 

ValueError: [Errno 110] Operation timed out
\end{verbatim}


\subsubsection{Conexión}
\verb"rm.list_resources()"
\verb"(u'ASRL/dev/ttyS1::INSTR', u'ASRL/dev/ttyS0::INSTR', u'ASRL/dev/ttyS3::INSTR', u'ASRL/dev/ttyS2::INSTR')"


¿Donde está conectado el TDS2002B?
\begin{verbatim}
dmesg | tail -20
[ 3096.816414] usb 2-1: USB disconnect, device number 4
[ 3097.424276] usb 2-1: new full-speed USB device number 5 using uhci_hcd
[ 3097.600185] usb 2-1: New USB device found, idVendor=0699, idProduct=0364
[ 3097.600191] usb 2-1: New USB device strings: Mfr=1, Product=2, SerialNumber=3
[ 3097.600197] usb 2-1: Product: Tektronix TDS2002B
[ 3097.600200] usb 2-1: Manufacturer: Tektronix, Inc.
[ 3097.600204] usb 2-1: SerialNumber: C101891
\end{verbatim}
¿Será el ttySm?




\section{Python2}\label{Python2}

\subsection{PyVISA}
\verb"https://pyvisa.readthedocs.org/en/master/index.html"

pip
    descargo get-pip.py de https://pip.pypa.io/en/stable/installing.html
    sudo -E python get-pip.py % -E requerido para mantener configuración proxy en el environment que no funcionó para root usando sudo simple

PyVISA
    sudo -E pip install -U pyvisa

Tengo el problema de que NI-Visa es de 32bit,
-> PyVISA-py
    http://python-in-the-lab.blogspot.com.ar/2014/10/communicating-with-instruments-using.html
    sudo -E pip install pyvisa-py

-> pyserial
    sudo -E pip install pyserial

No encontraba ningún usb, pues faltaba el backend de python para usb
->  PyUSB
    sudo -E pip install PyUSB
Ahora causa un error cada vez que esta el tek conectado

-> python-usbtmc
    sudo -E pip install python-usbtmc
No cambió nada

ERROR semi-corregido: Era un problema de permisos
    Probé el código que usa usbtmc de \verb"http://www.cibomahto.com/2010/04/controlling-a-rigol-oscilloscope-using-linux-and-python/"
    Y solo función bajo sudo -> Ídem. con test de PyVISA bajo sudo que arrojó como devices:
        (u'ASRL/dev/ttyS1::INSTR', u'ASRL/dev/ttyS0::INSTR', u'ASRL/dev/ttyS3::INSTR', u'ASRL/dev/ttyS2::INSTR', u'USB0::1689::868::C101891::0::INSTR')

ERROR corregido: Método self.write()
TypeError: Struct() argument 1 must be string, not unicode
Se corrige con python 2.7 mas nuevo: https://github.com/hgrecco/pyvisa-py/issues/6
Instalo la versión de Python de Jessie (ver \ref{testing})

ERROR corregido: Método self.write()
AttributeError: /lib/x86_64-linux-gnu/libusb-1.0.so.0: undefined symbol: libusb_strerror
sudo apt-get -t wheezy-backports install libusb-1.0-0

ERROR: Método self.write()
ValueError: [Errno 110] Operation timed out


\end{document}

